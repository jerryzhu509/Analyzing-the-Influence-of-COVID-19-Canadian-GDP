
\documentclass{article}  
\usepackage{apacite}  
\usepackage{amsmath}  
\usepackage[utf8]{inputenc}  
\usepackage[margin=0.75in]{geometry}  
\usepackage{array}  
\usepackage{graphicx}
\usepackage{setspace}  
\usepackage{indentfirst}   
\onehalfspacing  
\usepackage[utf8]{inputenc}  
\usepackage[english]{babel}  
\setlength{\parindent}{4em}  
\setlength{\parskip}{1em}  
\usepackage[shortlabels]{enumitem}  
  
  
\title{CSC110 Project Proposal: Predicting Canadian GDP During Covid-19}  
\author{Shu Fan Nicholas Au, Jack Sun, Jerry Zhu}  
\date{Friday, November 5, 2021}  
  
\begin{document}  
\maketitle  
  
\section{Introduction}  
Gross Domestic Product (GDP), whose official definition is the total monetary or market value of all the finished goods and services produced within a country’s borders in a specific time period, is the most commonly used indicator tracking the health of an economy (Kramer, 2021). It is a measurement of the aggregate economic activity, so not surprisingly, reduced economic activity during the pandemic has negatively impacted Canadian GDP. 

In fact, the Canadian GDP growth rate in 2020 was not only negative for the first time since the Financial Crisis of 2008, it decreased by a staggering 5.4\% in 2020 in contrast to a yearly decrease of 2.9\% in 2009 as a result of the Financial Crisis (Government of Canada, 2021). Given this unprecedented situation, we wish to understand whether the relevant economic indicators today can convincingly predict the GDP as well as it has in the past, or whether they fail to predict the GDP today because the pandemic has created an economic situation significantly different from anything before. Specifically, we consider, for quarterly periods from 2008 to the end of 2019 (just before COVID-19 hits in Canada), the following economic indicators:	 
  
\begin{enumerate}[topsep=0pt]  
  \itemsep0em  
  \item Balance of Payment  
  \item USD-CAD Exchange Rate  
  \item Bank of Canada Overnight Rate  
  \item Inflation Rate  
  \item 10-Year Canadian Government Bond Yield  
  \item Unemployment Rate  
  \item S\&P/TSX Composite Index  
  \item Population  
\end{enumerate}  
  
Of these, a study from Vignan University established that 1 and 2 significantly predict GDP (Divya & Devi, 2014). We felt that the rest are plausibly relevant. Further, some of these indicators warrant an explanation. The balance of payments is the record of all international financial transactions made by the residents of a country (Heakal, 2021). The Bank of Canada overnight rate is the interest rate at which other banks borrow from it. The inflation rate measures the rate at which prices are rising. We calculate the inflation rate through the Consumer Price Index, which measures the average prices over time for fixed goods. The 10-year government bond yield is the interest rate that the Canadian government pays to borrow money for 10 years. Finally, we consider the value of the S&P/TSX Composite Index, which tracks the performance of the largest companies listed on Canada's primary stock exchange, the Toronto Stock Exchange (TSX). Concisely, we ask: \textbf{How well can the changes to Canadian GDP Growth Rate caused by COVID-19 be predicted?}  
  
  
\section{Dataset Description}  
We are using datasets for the eight Canadian economic indicators and the Canadian GDP from 2008 - Q2/June 2021. Some of the datasets might also include data that are outside this date range, but we will only extract data that are in this range. The table below shows details of our datasets. Each dataset contains data either per day, per month or per quarter. We will only use data stored in the variable(s) described in “Variable to Use” (table) as well as their corresponding date.
\begin{center}  
  
\begin{tabular}{ | m{3cm} | m{5cm}| m{3cm} | m{2.5cm} | }   
  \hline  
  \textbf{Economic Indicators / GDP} & \textbf{Dataset File} & \textbf{Variable(s) to Use} & \textbf{Source}\\  
  \hline\hline  
  Canadian GDP (2008-2021) & GDP.csv & “Gross domestic product at market prices” at row 43 & Statistics Canada \\  
  \hline  
  Balance of Payment (2008-2021)& Balance\_of\_Payment,csv & “Total investment income, receipts” at row 12 & Statistics Canada \\  
  \hline  
  USD-CAD Exchange Rate (2008-2021)& USD-CAD\_Exchange\_Rate.csv & “Price” at column B & Investing.com Canada  \\  
  \hline  
  Bank of Canada Overnight Rate (2011-2021)& Bank\_of\_Canada\_Overnight Rate.csv & “V390779” at column B & Bank of Canada\\  
  \hline  
  Bank of Canada Overnight Rate (2008-2010)& Bank\_of\_Canada\_Overnight\_Rate
  \_2.csv & “Overnight money market financing” at row 12 & Statistics Canada\\  
  \hline  
  Inflation Rate (2008-2021)& Inflation\_Rate.csv & “INDINF\_CPI\_M” at column B & Bank of Canada\\  
  \hline  
  10-Year Canadian Government Bond Yield (2008-2021)& 10\_Year\_Canadian\_ Government\_Bond\_Yield.csv & “BD.CDN.10YR .DQ.YLD” at column F & Bank of Canada\\  
  \hline  
  Unemployment Rate (2008-2020)& Unemployment\_Rate.csv & “Canada” at row 14 & Statistics Canada \\ 
  \hline  
  Unemployment Rate (2021)& Unemployment\_Rate\_2.csv & “Unemployment rate 13” at row 28 & Statistics Canada \\  
  \hline  
  S\&P/TSX Composite Index (2008-2021)& S\_PTSX\_Composite\_Index.csv & “Standard and Poor's/Toronto Stock Exchange Composite Index, close” at row 14 & Statistics Canada \\  
  \hline  
  Population (2008-2021)& Population.csv & “Canada” at row 11 & Statistics Canada \\  
  \hline  
    
\end{tabular}  
\end{center}  
  
\section{Computational Overview}  
We constructed a machine learning model to predict quarterly GDP growth under the effect of Covid-19 through the 8 indicators. We train the model with data from 2008 to 2017, and have it predict the GDP growth rate by inputting economical indicators for 2018-2021. 2018-2019 is before covid period and 2020-2021 is during covid period. The predicted and actual GDP growth rate will be compared in their corresponding time period. We will compare the accuracy of the model predicting GDP growth rate before covid (2018-2019) and after covid (2020-2021) to make our final conclusion. The reason for selecting 2018-2019 is because this data wasn’t used as data training for the model, same as 2020-2021. 

There are two major steps in our program, data extraction and computing the model. 

In the step of data extraction, the function loading\_data() in reading\_data.py helps us to extract all the useful data into a structure that is ready for us to easily implement it into the model. 

The loading\_data() function used a lot of other helper functions. A few major ones will be listed here. We used the functions load\_horizontal\_data() and load\_vertical\_data() to systematically filter out specific values of the data that we need (the value’s corresponding variables are described in the above table) and transform it into a standard data structure (list[tuple[datetime.date, float]]). As some of the datasets have data per day, while others have data per week and per quarter. We implement functions change\_day\_to\_month() and change\_month\_to\_quarter() to convert all data into per quarter frequency. 

For the part of computing the model. We first convert the data from loaded\_data() into a DataFrame - a data type in the pandas library conducive for machine learning using scikit-learn. We process it as follows: add "dates" as a separate column, strip the dates from the indicator value, and fill in missing null values with official figures on the given date. We also compute "gdp\_growth" as follows: set the first row to be 0\%. For an arbitrary row i >1, the gdp growth is $\frac{gdp[i]-gdp[i-1]}{gdp[i-1]}*100$\%. Adding it to the DataFrame finishes the processing part.

With the processed DataFrame, we plot a correlation matrix of all the variables using Plotly and histograms of the columns using matplotlib, which allows us to make useful observations about the data to select the best pre-processing transformations. In particular, we conclude that transforming x with StandardScaler() and PCA() will benefit the model. We highlight these observations in 6.1.

To train the machine learning model, we use only the data from 2009-2017. We split the data into an array of independent variable values (x) and an array of the dependent variable(GDP) value (y), both of type np.ndarray. We then split x and y into training and testing sets using the train\_test\_split method from sklearn.model\_selection. Since the number of rows is low, the model is prone to overfitting; that is, the state of the model fitting the training data so tightly that it does not generalize in predictions. Therefore, we train our model on only 55\% of the data and use 45\% of the data to test it. Now, based on the aforementioned observations made on the graphs, we fit StandardScaler and PCA on the training set of the indicators to then use them to transform the entire indicator set.
 
Our candidate machine learning models are 'Elastic Net', 'Linear Support Vector Regressor', 'Random Forest Regressor', 'Gradient Boosting Regressor', and 'Extra Trees Regressor'. To select the best candidate, we use 5-fold cross validation on all the candidates with the help of the cross\_val\_score() method from sklearn.model\_selection. It splits x and y into 5 "folds" and calculates the Mean Squared Error (MSE) of the particular regressor on the "fold." Then, the method returns an np.ndarray of the errors. We then compute the Root Mean Squared Error (RMSE) with the average error, selecting the model that gives the lowest RMSE.

We find that ElasticNet gives the lowest RMSE. To optimize its performance, we perform hyperparameter tuning on it with the help of GridSearchCV method from sklearn.model_selection. Notable parameters to tune are "alpha", "l1_ratio", and "tol" (Brownlee, 2020). Using the fine-tuned model, we then make predictions on the GDP using indicator values from the entire DataFrame. In the function visualize_predictions() from model.py, we add "predicted_gdp" as a column to the DataFrame. Then, we plot a graph using matplotlib to visualize the prediction_gdp against the gdp on the vertical axis over the date on the horizontal axis (Figure 2). Using "predicted_gdp", we compute "predicted_gdp_growth" to visualize the prediction_gdp_growth against gdp_growth on the vertical axis over the date on the horizontal axis (Figure 3). Finally, we print the RMSE of the model on both gdp and gdp_growth on notable time intervals: 2008-2017 (training set), 2018-2019 (pre-covid), and 2020-2021 (during covid). The graphs and printed statistics allow us to draw interpretations in 6.1.


\section{Instructions for obtaining data sets and running your program}  
As our datasets aren’t that large and the Canda.ca has broken down again recently, we don’t have a direct link for downloading the csv. All of our datasets are uploaded on Markus. 

After running the main.py file, some values and several figures will be seen. 

Under the python console, the root mean square error (RMSE) of our model on the GDP and GDP growth rate in their corresponding date range are presented.


\includegraphics[width=15cm, height=4cm]{Screen Shot 2021-12-13 at 5.07.45 PM.png}


Besides, there will be a web page pop up in the browser. There is a correlation matrix representing the coefficient correlation between economic indicators and GDP. You may move your mouse to the box to see each of the specific values in the table. We can focus on reading the 3rd row, which focuses on describing the GDP. 

\includegraphics[width=15cm, height=10cm]{Screen Shot 2021-12-13 at 5.08.00 PM.png}


There will also be 3 figures popping up. Figure 1 is a histogram of the economic indicators and the GDP. Figure 2 shows the comparison between the predicted gdp and the actual gdp. Figure 3 shows the comparison between the predicted gdp growth and the actual gdp. For figure 2 and 3, you may move your mouse to a specific point to see the specific x and y coordinate of that point. 


\includegraphics[width=15cm, height=8cm]{Screen Shot 2021-12-13 at 5.08.52 PM.png}


\includegraphics[width=15cm, height=6cm]{Screen Shot 2021-12-13 at 5.09.07 PM.png}


\section{Changes in Project Plan}  
We didn’t make notable changes. We only changed some datasets that we are using. 

\section{Discussion section}  
\subsection{Results Interpretations}
From the correlation matrix, we observe that while indicators like "bond_yield" and "population" highly correlate with the dependent variable (GDP), many indicators highly correlate with each other as well; that is, there exists multicollinearity in the data. This means the model would be prone to overfitting. To remedy this, we apply Principal Component Analysis (PCA) and StandardScaler (Frost, 2021). The histograms give a sense of the distribution of the data. We make the observation that the indicators do not have significant outliers. As such, using StandardScaler() to scale the indicators is a logical choice (scikit-learn developers).
	
From the values printed in the python console, we could see that the RMSE of our model on gdp growth rate before covid times (2018-2019) are pretty decent with 0.505\%. However, with a RMSE of 4.25\% during covid times (2020-2021) is not that decent. By looking at the figure 3, we indeed see that despite the model accurately predicting the first two quarters of 2020, it doesn’t do a great job afterwards and makes a lot of wrong predictions, such as predicting a drastic increase while there is actually a drastic drop in GDP growth rate. This could be seen in figure 2, in which most of the predicted GDP are a lot lower than the actual GDP after 2020. 
	
The fact that the model is doing a good job predicting GDP before covid and did a bad job in predicting after covid shows that the economical indicators might not be that effective to predict changes to Canadian GDP Growth Rate caused by COVID-19. 
	
	
\subsection{limitations}
The different precision level and time interval of data sets are limitations. The overnight rate, inflation rate and bond yield indicators are recorded in days, which are very frequent. However, because data sets like GDP and population are updated 1 time per quarter, we have to unify the frequency of all 9 data sets to be 1 time per quarter by taking the average. This indeed keeps time consistency of data, but the number of available data is decreased to 54. Consequently, there are less available data sources for models to learn, which potentially undermines performance of models. Besides, the precision level of two data sets of overnight rate is inconsistent. Bank_of_Canada_Overnight_Rate_2.csv has a precision level 0.0001, while Bank_of_Canada_Overnight_Rate.csv only has 0.25. Such precision difference makes data inconsistent, which decreases the accuracy of the model.

\subsection{further exploration}
In this project, we demonstrated that the Covid-19 does undermine the accuracy of the model to predict GDP growth. The reason for inaccuracy is that we intentionally neglected Covid-19 indicators, such as number of active cases and days of quarantine. These indicators can negatively impact GDP growth, because Covid-19 kills labour forces and impedes staff-intensive industries. Next step, we will build an upgraded model to predict GDP growth, which inherits the structure of the original model and adds Covid-related variables. We will explore to what extent the Canadian GDP Growth Rate caused by COVID-19 be predicted in optimum.


\section*{References} 
Bank of Canada. (n.d.). Canadian interest rates and monetary policy variables: 10-year lookup. Retrieved November 5, 2021,\\ from https://www.bankofcanada.ca/rates/interest-rates/canadian-interest-rates/.

Bank of Canada. (n.d.). Inflation: Definitions, graphs and data. Retrieved November 5, 2021,\\ from https://www.bankofcanada.ca/rates/indicators/capacity-and-inflation-pressures/inflation/.

Bank of Canada. (n.d.). Selected bond yields. Retrieved November 5, 2021,\\ from https://www.bankofcanada.ca/rates/interest-rates/canadian-bonds/.

Brownlee, J. (2020, June 11). How to develop elastic net regression models in Python. Machine Learning Mastery. Retrieved December 13, 2021,\\ from https://machinelearningmastery.com/elastic-net-regression-in-python/. 

Chen, J. (2021, August 11). What is the overnight rate? Investopedia. Retrieved November 4, 2021,\\ from https://www.investopedia.com/terms/o/overnightrate.asp. 

Chen, J. (2021, September 21). Treasury yield. Investopedia. Retrieved November 4, 2021,\\ from https://www.investopedia.com/terms/t/treasury-yield.asp. 

D’monte, D. D. (2020, September 8). Predicting the nominal GDP using economic indicators: A data science approach. Medium. Retrieved November 4, 2021,\\ from https://towardsdatascience.com/predicting-the-nominal-gdp-using-economic-indicators-a-data-science-approach-7c56cded782. 

Divya, K. H., Devi, V. R. (2014, July 14). A study on predictors of GDP: Early signals. Procedia Economics and Finance. Retrieved November 4, 2021,\\ from https://www.sciencedirect.com/science/article/pii/S2212567114002056. 

Fernando, J. (2021, September 8). What is the S&P/TSX Composite Index? Investopedia. Retrieved November 4, 2021,\\ from https://www.investopedia.com/terms/s/sp-tsx-composite-index.asp. 

Frost, J. (2021, September 24). Multicollinearity in regression analysis: Problems, detection, and solutions. Statistics By Jim. Retrieved December 13, 2021,\\ from https://statisticsbyjim.com/regression/multicollinearity-in-regression-analysis/. 

Government of Canada, Statistics Canada. (2021, December 3). Table 14-10-0287-01. Labour force characteristics, monthly, seasonally adjusted and trend-cycle, last 5 months. Retrieved December 10, 2021,\\ from https://www150.statcan.gc.ca/t1/tbl1/en/tv.action?pid=1410028701.

Government of Canada, Statistics Canada. (2021, December 9). Table 10-10-0139-01 Bank of Canada, money market and other interest rates. Retrieved December 9, 2021,\\ from https://www150.statcan.gc.ca/t1/tbl1/en/tv.action?pid=1010013901.

Government of Canada, Statistics Canada. (2021, January 8). Table 14-10-0294-01 Labour force characteristics by census metropolitan area, three-month moving average, seasonally adjusted and unadjusted, last 5 months, inactive. Retrieved November 5, 2021,\\ from https://www150.statcan.gc.ca/t1/tbl1/en/tv.action?pid=1410029401.

Government of Canada, Statistics Canada. (2021, November 1). Table 10-10-0125-01 Toronto Stock Exchange Statistics. Retrieved November 5, 2021,\\ from https://www150.statcan.gc.ca/t1/tbl1/en/tv.action?pid=1010012501.

Government of Canada, Statistics Canada. (2021, November 29). Table 36-10-0018-01  Balance of international payments, current account, seasonally adjusted, quarterly (x 1,000,000) Retrieved December 9, 2021,\\ from https://www150.statcan.gc.ca/t1/tbl1/en/tv.action?pid=3610001801

Government of Canada, Statistics Canada. (2021, November 30). Table 36-10-0104-01  	Gross domestic product, expenditure-based, Canada, quarterly (x 1,000,000) Retrieved December 9, 2021,\\ from https://www150.statcan.gc.ca/t1/tbl1/en/tv.action?pid=3610010401

Government of Canada, Statistics Canada. (2021, September 29). Table 17-10-0009-01 Population Estimates, quarterly. Population estimates, quarterly. Retrieved November 5, 2021,\\ from https://www150.statcan.gc.ca/t1/tbl1/en/tv.action?pid=1710000901.

Heakal, R. (2021, September 8). What is the balance of payments? Investopedia. Retrieved November 4, 2021,\\ from https://www.investopedia.com/insights/what-is-the-balance-of-payments/. 

Kramer, L. (2021, July 27). What is GDP and why is it so important to economists and investors? Investopedia. Retrieved November 4, 2021,\\ from https://www.investopedia.com/ask/answers/what-is-gdp-why-its-important-to-economists-investors/. 

Learn. scikit. (n.d.). Retrieved November 4, 2021,\\ from https://scikit-learn.org/stable/index.html#. 

scikit-learn developers. (n.d.). Compare the effect of different scalers on data with outliers. scikit. Retrieved December 13, 2021,\\ from https://scikit-learn.org/stable/auto\_examples/preprocessing/plot\_all\_scaling.html. 

USD CAD Historical Data. Investing.com Canada. (n.d.). Retrieved November 5, 2021,\\ from https://ca.investing.com/currencies/usd-cad-historical-data

\end{document} 